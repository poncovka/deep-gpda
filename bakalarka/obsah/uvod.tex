
%%%%%%%%%%%%%%%%%%%%%%%%%%%%%%%%%%%%%%%%%%%%%%%%%%%%%%%%%%%%%%%%%%%%%%%%%%
\chapter{Úvod}
%TODO Přidat informace o redukcich?

Od vzniku teorie formálních jazyků byl kladen důraz na bezkontextové jazyky. Ve většině případů jsou jejich možnosti postačující, a proto je dnes tato problematika velmi dobře prozkoumaná a zdokumentovaná. Nicméně \uv{svět je kontextový} \cite{Dassow:RegulatedRewriting}. V~oblastech, jako jsou přirozené jazyky, vývojová biologie nebo například sémantika programovacích jazyků, je kontext důležitý. Na druhou stanu kontextové a vyšší jazyky jsou příliš mocné a špatně se s~nimi pracuje. Řešením bylo spojit \uv{jednoduchost a krásu bezkontextových gramatik se sílou kontextových} \cite{Dassow:RegulatedRewriting} a zavést tzv. řízené gramatiky. Mnohé z~těchto gramatik definují jazykovou rodinu ležící mezi rodinou bezkontextových jazyků a jazyků kontextových.

V~důsledku tohoto vývoje se začaly zavádět automaty ekvivalentní těmto gramatikám. Příkladem budiž hluboký zásobníkový automat \cite{Meduna:DeepPDA} korespondující s~n-limited stavovou gramatikou a hluboký zásobníkový automat konečného indexu \cite{Meduna:FinitelyDeepPDA} ekvivalentní s~maticovou gramatikou konečného indexu. V~této práci zkoumám tyto dva modely a zavádím jejich nové modifikace. Hlavním kritériem je redukce. Zaměřuji se na snížení počtu stavů, nevstupních symbolů, případně zjednodušení pravidel a zkoumám dopad na sílu automatu.


V~kapitole \ref{kap_deep_pda_fin} zavádím hluboký zásobníkový automat konečného indexu s~jedním nevstupním symbolem a dokazuji jeho ekvivalenci s~programovými gramatikami konečného indexu. Způsob, jakým tento model pracuje, aplikuji na hluboký zásobníkový automat v~kapitole \ref{kap_pda_NF}. Výsledkem je zavedení normální formy hlubokého zásobníkového automatu a algoritmus pro převod automatů do této formy. V~následující kapitole \ref{kap_pda_gen} představuji zobecněný hluboký zásobníkový automat. Zobecnění spočívá v~expanzi nejvrchnějšího symbolu na zásobníku, který lze v~daném stavu expandovat. Původní hluboký zásobníkový automat měl hloubku expanze definovanou v~pravidle. Dále zkoumám sílu zobecněného automatu vzhledem ke stavovým gramatikám a pokouším se minimalizovat počet nevstupních symbolů a stavů. Způsob této redukce demonstruji v~aplikaci, jejíž návrh a implementaci popisuji v~kapitole \ref{kap_aplikace}. Jedná se o~konzolovou aplikaci, která na standardní výstup vypisuje zredukovaný zobecněný zásobníkový automat dle nastavených parametrů.


