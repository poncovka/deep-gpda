
%%%%%%%%%%%%%%%%%%%%%%%%%%%%%%%%%%%%%%%%%%%%%%%%%%%%%%%%%%%%%%%%%%%%%%%%%% 
\chapter{Aplikace \texttt{gdeep\_pda}} \label{kap_aplikace}

V~této kapitole popisuji návrh a implementaci konzolové aplikace pro redukci zobecněných hlubokých zásobníkových automatů a syntaktickou analýzu řetězců, které tyto automaty přijímají. Na aplikaci demonstruji implementaci algoritmů \ref{alg_gen_deep_pda_nonterm} a \ref{alg_gen_deep_pda_state} navržených v~kapitole \ref{kap_pda_gen}.


\section{Návrh aplikace a formátu vstupního souboru}

Při návrhu aplikace jsem vycházela z~kapitoly \ref{kap_pda_gen} věnované zobecněným hlubokým zásobníkovým automatům. Dále bylo třeba navrhnout základní komponenty aplikace, postup pro syntaktickou analýzu řetězce a formát zápisu automatu. 

\subsubsection{Rozvržení aplikace}\label{kap_aplikace_rozvrzeni}

Aplikace se skládá z~několika částí. Práci se zásobníkovým automatem umožňují moduly \texttt{automaton}, \texttt{parser}, \texttt{symbol\_reduction} a \texttt{state\_reduction}. 
Modul \texttt{automaton} obsahuje třídu reprezentující zobecněný hluboký zásobníkový automat, \texttt{parser} zprostředkovává načtení zápisu automatu z~řetězce a \texttt{symbol\_reduction} spolu se \texttt{state\_reduction} slouží k~redukci symbolů nebo stavů automatu.
Činnost aplikace pak zajišťují moduly \texttt{application}, \texttt{error} a \texttt{library}, kde \texttt{application} umožňuje běh aplikace, \texttt{error} definuje chybové stavy a \texttt{library} slouží jako knihovna funkcí.

\subsubsection{Syntaktická analýza řetězce}\label{kap_aplikace_analyza}

Zobecněné hluboké zásobníkové automaty jsou nedeterministické, což je třeba u~syntaktické analýzy zohlednit. Vzhledem k~demonstračním účelům aplikace jsem se rozhodla použít metodu shora dolů s~návratem, která je z~hlediska časové složitosti neefektivní, ale jednoduchá na implementaci.

Automat postupně provádí jednotlivé kroky derivace tak, že na nejvrchnější možný nevstupní symbol aplikuje první nalezené pravidlo. Pokud je na vrcholu zásobníku vstupní symbol, provede se pop operace. Jestliže tuto operaci nelze provést, automat se vrátí do předchozí konfigurace a pokusí se aplikovat jiné pravidlo. V~případě neúspěchu je opět proveden návrat. Pokud se automat vrátí za výchozí konfiguraci, pak analyzovaný řetězec není řetězcem jazyka přijímaného tímto automatem. V~opačném případě algoritmus skončí, jakmile je přečten celý vstupní řetězec, automat je v~konečném stavu a zásobník je prázdný.

Automat prochází derivační strom do hloubky. Je proto vhodné omezit hloubku stromu, jinak by analýza řetězce mohla vést k~zacyklení aplikace. Pro ukládání konfigurací a obnovení konfigurace předcházející lze použít zásobník. Po každém kroku automatu se na tento zásobník vloží aktuální konfigurace a pořadové číslo aplikovaného pravidla. Při neúspěchu se automat nastaví podle konfigurace na vrcholu zásobníku. Po úspěšné analýze lze pomocí tohoto zásobníku zrekonstruovat jednotlivé kroky derivace.

\subsubsection{Formát zápisu automatu}\label{kap_aplikace_format}

Zápis zobecněného zásobníkového automatu vychází z~jeho definice. Hlavním kritériem formátování bylo, aby výstupem programu byl automat, který se může použít jako vstup. Zároveň v~souvislosti s~redukcí jsem musela umožnit vytvářet složitější zápisy stavů a nevstupních symbolů, aby se daly použít k~uchovávání nezbytných informací. Příklad formátování je k~dispozici v~\ref{example_format}. Zevrubnější popis je součástí manuálu k~aplikaci.

\begin{Example} \label{example_format}
Ukázka zápisu zobecněného hlubokého automatu pro aplikaci \texttt{gdeep\_pda}.

\begin{verbatim}
// Hluboký zásobníkový automat:
(
    // množina stavů
    { <s1>, <s2>, <s3>, <s4>, <s5> },
    // množina vstupních symbolů
    { a, b, c },
    // množina zásobníkových symbolů
    { A, B, C, a, b, c },
    // množina pravidel
    { 
        <s1> A -> <s2> a A b B c C,
        <s2> A -> <s3> a A,
        <s3> B -> <s4> b B,
        <s4> C -> <s2> c C,
        <s2> A -> <s5>,            // epsilon pravidlo
        <s5> B -> <s5> '',         // epsilon pravidlo
        <s5> C -> <s5> '' '' ''    // epsilon pravidlo
    },
    <s1>,     // počáteční stav
    A,        // počáteční symbol
    { <s5> }  // množina koncových stavů	
)
\end{verbatim}
\end{Example}

\section{Implementace aplikace}

Aplikaci \texttt{gdeep\_pda} jsem implementovala v~jazyce Python verze 3.2 jako spustitelný balík. Cílovou platformou byly operační systémy unixového typu. Na jiných platformách aplikace nebyla testována.

\subsubsection{Činnost aplikace}

Po spuštění je zavolána funkce \texttt{main()} z~modulu \texttt{application}. Ta zpracuje parametry, načte vstup do řetězce a řetězec zpracuje pomocí parseru typu \texttt{GDPParser}. Vstupní automat se pak formátovaně vypíše na výstup, nebo se získá jeho zredukovaná varianta pomocí objektu třídy \texttt{StateReduction}, případně \texttt{SymbolReduction}, a vypíše se zredukovaný automat. Modul \texttt{library} poskytuje funkce pro výpis nápovědy, zpracování parametrů a práci se vstupy a výstupy.


\subsubsection{Model zobecněného hlubokého zásobníkového automatu}

Třída \texttt{GDP} z~modulu \texttt{automaton} reprezentuje zobecněný hluboký zásobníkový automat. Komponenty automatu jsou implementované jako instanční proměnné \texttt{Q}, \texttt{Sigma}, \texttt{Gamma}, \texttt{R}, \texttt{s}, \texttt{S} a \texttt{F}, jejichž pojmenování odpovídá definici \ref{def_pda_gen}. 

Proměnné \texttt{Q}, \texttt{Sigma}, \texttt{Gamma}, \texttt{F} jsou množiny řetězců, \texttt{s}, \texttt{S} jsou řetězce a \texttt{R} je množina uspořádaných čtveřic \texttt{(q, A, p, v)}, kde \texttt{q}, \texttt{A}, \texttt{p} jsou řetězce, \texttt{v} je n-tice řetězců a čtveřice je reprezentací pravidla $qA \rightarrow pv$. 
K~snazšímu sestavování komplikovanějších pravidel slouží třída \texttt{GDP\_rule}, kde \texttt{q}, \texttt{A}, \texttt{p}, \texttt{v} jsou proměnné instance a metoda \texttt{get()} vrací uspořádanou čtveřici.

Metoda \texttt{validate()} kontroluje sémantickou správnost automatu a v~případě chyby vyvolá výjimku \texttt{EPDA}. Metoda \texttt{serialize()} vrací řetězec s~formátovaným zápisem automatu.

\subsubsection{Parsování automatu z~řetězce}

Původně jsem vstupní řetězec zpracovávala výhradně pomocí regulárních výrazů, ale pro obsáhlejší automaty tento postup výrazně zpomaloval aplikaci. Dále jsem zvažovala použít konečný automat pro lexikální analýzu a syntaxi analyzovat metodou shora dolů, ale takové řešení by bylo velmi robustní a neumožňovalo jednoduše měnit formát zápisu. Nakonec jsem oba postupy zkombinovala. Stavy a symboly zpracovávám regulárními výrazy, automat a pravidla konečným automatem. Konečný automat je pak řízen pravidly, která jsou specifikovaná formou zjednodušených regulárních výrazů. Činnost parseru je tak nezávislá na formátu zápisu automatu a bylo by snadné umožnit~analýzu dalších typů automatů v~různých formátech.

Zpracování vstupu zajišťuje třída \texttt{GDPParser} z~modulu \texttt{parser}. 
Při inicializaci se nastaví vzory (regulární výrazy a pravidla) a analýza řetězce se spustí zavoláním metody \texttt{run()}. Parsování zajišťují funkce \texttt{match()}, \texttt{matchStr()}, \texttt{matchItem()}, \texttt{matchList()} a \texttt{matchGroup()}, které vrací dvojici prvek, index. Prvek je nalezený řetězec nebo seznam prvků, index označuje pozici v~řetězci pro další parsování. Metoda \texttt{match()} pak slouží k~aplikaci regulárního výrazu, \texttt{matchStr()} porovnává řetězec, \texttt{matchItem()} zavolá podle typu vzoru odpovídající metodu, \texttt{matchList()} volá matchItem() pro všechny vzory v~předaném seznamu a \texttt{matchGroup()} cyklicky volá \texttt{matchItem()} pro analýzu položek prokládaných oddělovačem. Nakonec metoda \texttt{run()} vrátí načtený automat typu \texttt{GDP}, nebo vyvolá výjimku \texttt{EPDA}.

\subsubsection{Redukce nevstupních symbolů}

Třída \texttt{SymbolReduction} z~modulu \texttt{symbol\_reduction} umožňuje provádět redukci počtu nevstupních symbolů nad zobecněnými hlubokými zásobníkovými automaty. Metoda \texttt{run()} obdrží automat typu \texttt{GDP} a vrátí automat zkonstruovaný podle algoritmu \ref{alg_gen_deep_pda_nonterm} z~kapitoly \ref{kap_pda_gen}.
Metoda \texttt{getCodingFunction()} vrátí kódovací funkci pro symboly. Tato funkce je reprezentovaná datovým typem slovník, kde vstupem funkce je klíč a výstupem funkce hodnota. Následně jsou metodami \texttt{construct\_Q()} a \texttt{construct\_R()} sestaveny množiny stavů a pravidel a je zkonstruován nový automat. Jeho správnost se ověřuje metodou \texttt{validate()} z~třídy \texttt{GDP}.

\subsubsection{Redukce stavů}

K~redukci počtu stavů slouží třída \texttt{StateReduction} z~modulu \texttt{state\_reduction}. Konstrukci automatu se třemi stavy provádí metoda \texttt{run()} podle algoritmu \ref{alg_gen_deep_pda_state} z~kapitoly \ref{kap_pda_gen}. Metoda \texttt{getIndexFunction()} seřadí stavy automatu podle abecedy a vrátí funkci pro určení pozice stavu v~tomto uspořádání. Metoda \texttt{getStateFunction()} pak definuje a vrátí funkci, která pro každý stav automatu vrací stav následující a pro poslední stav stav první. Poté metoda \texttt{construct\_Gamma()} zkonstruuje množinu zásobníkových symbolů a metoda \texttt{construct\_R()} sestaví množinu pravidel. Nakonec je vytvořen a inicializován výsledný automat a ověřena jeho správnost.

\subsubsection{Analýza vstupního řetězce}
Syntaktickou analýzu řetězce umožňuje třída \texttt{GDP} z~modulu \texttt{automaton}. Metoda \texttt{analyze()} hledá derivaci řetězce pomocí postupu navrhovaném v~kapitole \ref{kap_aplikace_analyza}. Aktuální konfiguraci automatu lze získat voláním \texttt{getConfiguration()}, \texttt{saveDerivation()} a \texttt{loadDerivation()} umožňují ukládat konfigurace na zásobník a obnovit stav automatu podle poslední uložené konfigurace. Pokud analýza proběhne úspěšně, pak funkce \texttt{tableprint()} z~modulu \texttt{library} převede kroky derivace do řetězce a výsledek se vypíše na výstup. Hloubku derivace lze nastavit parametrem pro maximální počet kroků.

\subsubsection{Ošetření chybových stavů}

V~modulu \texttt{error} jsou definované výjimky \texttt{Error}, \texttt{EPARAM}, \texttt{EIO} a \texttt{EPDA}. \texttt{Error} je základní třída, kterou dědí ostatní výjimky. \texttt{EPARAM} indikuje chybu v~parametrech, \texttt{EIO} chybu při práci se soubory a \texttt{EPDA} označuje chybu v~zásobníkovém automatu. Pokud je vyvolána výjimka, na standardní chybový výstup se vypíše chybová hláška a aplikace skončí s~odpovídající návratovou hodnotou.

\section{Testování}

Pro účely testování jsem použila framework \texttt{unittest} a vytvořila několik testovacích případů k~otestování hlavních komponent aplikace: zpracování parametrů, zpracování vstupu, analýzu řetězce a redukci automatu. Všechny testy jsou součástí balíku \texttt{test} a spustitelné příkazem \texttt{python3 test.py}.

\section{Spuštění aplikace}

Aplikace \texttt{gdeep\_pda} pro svůj běh vyžaduje Python verze 3.2. Manuál s~podrobnějšími informacemi je k~dispozici v~souboru \texttt{MANUAL}. Aplikaci lze spustit příkazem:

\begin{verbatim}
python3 gdeep_pda.py [-h|--help] [--input=filename] [--output=filename] 
                     [--reduce-states] [--reduce-symbols] 
                     [--analyze-string="s t r i n g"] [--max-steps=n]
\end{verbatim} 










%=========================================================================
