
%%%%%%%%%%%%%%%%%%%%%%%%%%%%%%%%%%%%%%%%%%%%%%%%%%%%%%%%%%%%%%%%%%%%%%%%%% 
\chapter{Aplikace \texttt{gdeep\_pda}} \label{kap_aplikace}

V~této kapitole popisuji návrh a implementaci konzolové aplikace pro redukci zobecněných hlubokých zásobníkových automatů. Na aplikaci demonstruji implementaci algoritmů \ref{alg_gen_deep_pda_nonterm} a \ref{alg_gen_deep_pda_state} navržených v~kapitole \ref{kap_pda_gen}.


\section{Návrh aplikace a formátu vstupního souboru}

%Aplikace \texttt{gdeep\_pda} by měla umožnit načíst vstupní automat ze souboru nebo standardního vstupu. Formát zápisu automatu je navržen v kapitole \ref{kap_aplikace_format}. Parsování vstupního řetězce je možné provést pomocí regulárních výrazů nebo prostřednictvím lexikální a syntaktické analýzy.  

Při návrhu aplikace jsem vycházela z~kapitoly věnované zobecněným hlubokým zásobníkovým automatům. Odkáži se tedy na kapitolu \ref{kap_pda_gen} a použité principy zde nebudu dále rozebírat. %Základní kostru aplikace popisuji v kapitole \ref{kap_aplikace_rozvrzeni} a popisu vstupního souboru se věnuji v kapitole \ref{kap_aplikace_format}.

\subsubsection{Rozvržení aplikace}\label{kap_aplikace_rozvrzeni}

Aplikace se skládá z~několika částí. Práci se zásobníkovým automatem umožňují moduly \texttt{automaton}, \texttt{parser}, \texttt{symbol\_reduction} a \texttt{state\_reduction}. 
Modul \texttt{automaton} obsahuje třídu reprezentující zobecněný hluboký zásobníkový automat, \texttt{parser} zprostředkovává načtení zápisu automatu z~řetězce a \texttt{symbol\_reduction} spolu se \texttt{state\_reduction} slouží k~redukci symbolů nebo stavů automatu.
Činnost aplikace pak zajišťují moduly \texttt{application}, \texttt{error} a \texttt{library}, kde \texttt{application} umožňuje běh aplikace, \texttt{error} definuje chybové stavy a \texttt{library} slouží jako knihovna funkcí.

\subsubsection{Syntaktická analýza řetězce}
DOPLNIT

\subsubsection{Formát zápisu automatu}\label{kap_aplikace_format}

%Zápis zobecněného zásobníkového automatu vychází z~jeho definice. Automat je uzavřen v~kulatých závorkách, množiny ve složených a jednotlivé komponenty jsou oddělené čárkami a libovolným počtem bílých znaků. Dále prvky pravidla jsou oddělené bílými znaky a přechod se zapisuje \texttt{->}. 

%Stavy a symboly zásobníkové abecedy se smí skládat pouze z~písmen a číslic. Výjimkou jsou symboly vstupní abecedy uzavřené v~apostrofech a stavy a nevstupní symboly v~závorkách typu \texttt{<>}, \texttt{[]} nebo \texttt{()}. V~těchto případech mohou tyto komponenty obsahovat libovolné tisknutelné znaky s~výjimkou toho, který komponentu uzavírá. Stavy a nevstupní symboly navíc mohou obsahovat řetězec tisknutelných znaků v~apostrofech.


\begin{Example} \label{example_format}
Ukázka zápisu zobecněného hlubokého automatu pro aplikaci \texttt{gdeep\_pda}.

\begin{verbatim}
// Hluboký zásobníkový automat:
(
    // množina stavů
    { <s1>, <s2>, <s3>, <s4>, <s5> },
    // množina vstupních symbolů
    { a, b, c },
    // množina zásobníkových symbolů
    { A, B, C, a, b, c },
    // množina pravidel
    { 
        <s1> A -> <s2> a A b B c C,
        <s2> A -> <s3> a A,
        <s3> B -> <s4> b B,
        <s4> C -> <s2> c C,
        <s2> A -> <s5>,            // epsilon pravidlo
        <s5> B -> <s5> '',         // epsilon pravidlo
        <s5> C -> <s5> '' '' ''    // epsilon pravidlo
    },
    <s1>,     // počáteční stav
    A,      // počáteční symbol
    { <s5> }  // množina koncových stavů	
)
\end{verbatim}
\end{Example}

\section{Implementace aplikace}

Aplikaci \texttt{gdeep\_pda} jsem implementovala v~jazyce Python verze 3.2 jako spustitelný balík. Vzhledem k~vybranému programovacímu jazyku výsledná aplikace není závislá na platformě.


\subsubsection{Činnost aplikace}

Po spuštění aplikace je zavolána funkce \texttt{main()} z~modulu \texttt{application}. Ta zpracuje parametry, načte vstup do řetězce a řetězec zpracuje pomocí parseru typu \texttt{GDPParser}. Vstupní automat se pak formátovaně vypíše na výstup, nebo se získá jeho zredukovaná varianta pomocí objektu třídy \texttt{StateReduction}, případně \texttt{SymbolReduction}, a vypíše se zredukovaný automat. Modul \texttt{library} poskytuje funkce pro výpis nápovědy, zpracování parametrů a práci se vstupy a výstupy.


\subsubsection{Model zobecněného hlubokého zásobníkového automatu}

Třída \texttt{GDP} z~modulu \texttt{automaton} reprezentuje zobecněný hluboký zásobníkový automat. Komponenty automatu jsou implementované jako instanční proměnné \texttt{Q}, \texttt{Sigma}, \texttt{Gamma}, \texttt{R}, \texttt{s}, \texttt{S} a \texttt{F}, jejichž pojmenování odpovídá definici \ref{def_pda_gen}. 

Proměnné \texttt{Q}, \texttt{Sigma}, \texttt{Gamma}, \texttt{F} jsou množiny řetězců, \texttt{s}, \texttt{S} jsou řetězce a \texttt{R} je množina uspořádaných čtveřic \texttt{(q, A, p, v)}, kde \texttt{q}, \texttt{A}, \texttt{p} jsou řetězce, \texttt{v} je n-tice řetězců a čtveřice je reprezentací pravidla $qA \rightarrow pv$. 
K~snazšímu sestavování komplikovanějších pravidel slouží třída \texttt{GDP\_rule}, kde \texttt{q}, \texttt{A}, \texttt{p}, \texttt{v} jsou proměnné instance a metoda \texttt{get()} vrací uspořádanou čtveřici.

Metoda \texttt{validate()} kontroluje sémantickou správnost automatu a v~případě chyby vyvolá výjimku \texttt{EPDA}. Metoda \texttt{serialize()} vrací řetězec s~formátovaným zápisem automatu.

\subsubsection{Parsování automatu z~řetězce}

Původně jsem vstupní řetězec zpracovávala výhradně pomocí regulárních výrazů, ale pro obsáhlejší automaty tento postup výrazně zpomaloval aplikaci. Dále jsem zvažovala použít konečný automat pro lexikální analýzu a syntaxi analyzovat metodou shora dolů, ale takové řešení by bylo velmi robustní a neumožňovalo jednoduše měnit formát zápisu. Nakonec jsem oba postupy zkombinovala. Stavy a symboly zpracovávám regulárními výrazy, automat a pravidla konečným automatem. Konečný automat je pak řízen pravidly, která jsou specifikovaná formou zjednodušených regulárních výrazů. Činnost parseru je tak nezávislá na formátu zápisu automatu a bylo by snadné umožnit~analýzu dalších typů automatů v~různých formátech.

Zpracování vstupu zajišťuje třída \texttt{GDPParser} z~modulu \texttt{parser}. 
Při inicializaci se nastaví vzory (regulární výrazy a pravidla) a analýza řetezce se spustí zavoláním metody \texttt{run()}. Parsování zajišťují funkce \texttt{match()}, \texttt{matchStr()}, \texttt{matchItem()}, \texttt{matchList()} a \texttt{matchGroup()}, které vrací dvojici prvek, index. Prvek je nalezený řetězec nebo seznam prvků, index označuje pozici v~řetězci pro další parsování. Metoda \texttt{match()} pak slouží k~aplikaci regulárního výrazu, \texttt{matchStr()} porovnává řetězec, \texttt{matchItem()} zavolá podle typu vzoru odpovídající metodu, \texttt{matchList()} volá matchItem() pro všechny vzory v~předaném seznamu a \texttt{matchGroup()} cyklicky volá \texttt{matchItem()} pro analýzu položek prokládaných oddělovačem. Nakonec metoda \texttt{run()} vrátí načtený automat typu \texttt{GDP}, nebo vyvolá výjimku \texttt{EPDA}.

\subsubsection{Redukce nevstupních symbolů}

Třída \texttt{SymbolReduction} z~modulu \texttt{symbol\_reduction} umožňuje provádět redukci počtu nevstupních symbolů nad zobecněnými hlubokými zásobníkovými automaty. Metoda \texttt{run()} obdrží automat typu \texttt{GDP} a vrátí automat zkonstruovaný podle algoritmu \ref{alg_gen_deep_pda_nonterm} z~kapitoly \ref{kap_pda_gen}.
Metoda \texttt{getCodingFunction()} vrátí kódovací funkci pro symboly. Tato funkce je reprezentovaná datovým typem slovník, kde vstupem funkce je klíč a výstupem funkce hodnota. Následně jsou metodami \texttt{construct\_Q()} a \texttt{construct\_R()} sestaveny množiny stavů a pravidel a je zkonstruován nový automat. Jeho správnost se ověřuje metodou \texttt{validate()} z~třídy \texttt{GDP}.

\subsubsection{Redukce stavů}

K~redukci počtu stavů slouží třída \texttt{StateReduction} z~modulu \texttt{state\_reduction}. Konstrukci automatu se třemi stavy provádí metoda \texttt{run()} podle algoritmu \ref{alg_gen_deep_pda_state} z~kapitoly \ref{kap_pda_gen}. Metoda \texttt{getIndexFunction()} seřadí stavy automatu podle abecedy a vrátí funkci pro určení pozice stavu v~tomto uspořádání. Metoda \texttt{getStateFunction()} pak definuje a vrátí funkci, která pro každý stav automatu vrací stav následující a pro poslední stav stav první. Poté metoda \texttt{construct\_Gamma()} zkonstruuje množinu zásobníkových symbolů a metoda \texttt{construct\_R()} sestaví množinu pravidel. Nakonec je vytvořen a inicializován výsledný automat a ověřena jeho správnost.

\subsubsection{Analýza vstupního řetězce}
DOPSAT

\subsubsection{Ošetření chybových stavů}

V~modulu \texttt{error} jsou definované výjimky \texttt{Error}, \texttt{EPARAM}, \texttt{EIO} a \texttt{EPDA}. \texttt{Error} je základní třída, ze které dědí ostatní výjimky. \texttt{EPARAM} indikuje chybu v~parametrech, \texttt{EIO} chybu při práci se soubory a \texttt{EPDA} označuje chybu v~zásobníkovém automatu. Pokud je vyvolána výjimka, na standardní chybový výstup se vypíše chybová hláška a aplikace skončí s~odpovídající návratovou hodnotou.

\section{Testování}
DOPSAT

%=========================================================================
