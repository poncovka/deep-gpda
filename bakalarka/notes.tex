
%%%%%%%%%%%%%%%%%%%%%%%%%%%%%%%%%%%%%%%%%%%%%%%%%%%%%%%%%%%%%%%%%%%%%%%%%%%%%%% Definice
%V~této práci používám značení a pojmy z~teorie formálních jazyků zavedené například v~\cite{Meduna:Theory}. Dále vycházím z~definice maticové a programové gramatiky v~\cite{Dassow:RegulatedRewriting}.


%Pozn. Je třeba si uvědomit, že hluboký zásobníkový automat konečného indexu a gramatika konečního indexu se mají jinak definovanou svoji konečnost. Zatímco u hlubokého zásobníkového automatu nemůže nikdy dojít k tomu, že na zásobníku bude více než n nevstupních symbolů, gramatiky konečného indexu mohou expandovat řetězec s více než n neterminály. Pro každý řetězec terminálů však existuje taková sekvence derivačních kroků, která v žádném kroku neobsahuje více než n neterminálů.


%\begin{Def}
%Bezkontextová gramatika je dle \cite{doplnit} čtveřice $G = (V,T,P,S)$, kde V je úplná abeceda, T \subset V je abeceda terminálů. Nechť N = (V - T) je abeceda neterminálů, pak S \in N je počáteční symbol. P je konečná množina pravidel, která zapisujeme ve tvaru q: A \rightarrow v, kde A \in N, v \in V^* a q je označení pravidla.

%Nechť q: A \rightarrow v \in P, x,y \in V^*, pak bezkontextová gramatika G provede derivační krok z xAy do xvy podle pravidla q; píšeme xAy \Rightarrow xvy [q] nebo zjednodušeně xAy \Rightarrow xvy. Nechť q_1, q_2, \dots q_m \in P, m \ge 0, pak G může provést sekvenci kroků dle těchto pravidel; zápisem x {\Rigtarrow}^m y [q_1 q_2 \dots q_m]. Píšeme {\Rigtarrow}^+ pro libovolné m > 0 a {\Rigtarrow}^* pro m \ge 0. Gramatika G generuje jazyk L(G), pro kteý platí, že L(G) = {w \in T^* | S {\Rigtarrow}^* w}. Jazyk L je bezkontektový, právě když L = L(G).

%\end{Def}

%\begin{Def}
%Programová gramatika je čtveřice $G = (V,T,P,S)$, která vychází z bezkontektové gramatiky. P je konečná množina pravidel tvaru q:A \rigtarrow v, g(p), kde q:A \rigtarrow definujeme stejně jako v bezkotextové gramatice a q(p) je množina značení těch pravidel, která mohou být provedena v dalším derivačním kroku.

%TODO zavést sekvence derivačních kroků
%\end{Def}

% zavést index gramatiky

% zavést maticovou gramatiku

%%%%%%%%%%%%%%%%%%%%%%%%%%%%%%%%%%%%%%%%%%%%%%%%%%%%%%%%%%%%%%%%%%%%%%%%%%%%%%%%%

%=========================================================================
\section{Převod na normální formu}


\begin{Def}
Nechť $M = (Q,\Sigma,\Gamma, R, s, S , F)$ je hluboký zásobníkový automat
a $n$ je maximální hloubka expanze v~$M$.
Pak $M$ je v~normální formě, pokud každé pravidlo z~$R$ je v~jednom ze tvarů:

\begin{enumerate}
\renewcommand{\labelenumi}{(\roman{enumi})}
\item $mq\# \rightarrow pA$, kde $A \in (\Gamma - \Sigma)$, $m = n$, $p$, $q \in Q$
\item $mqA \rightarrow p\#$, kde $A \in (\Gamma - \Sigma)$, $1 \le m < n$, $p$, $q \in Q$
\item $mq\# \rightarrow p\#$, kde $1 \le m < n$, $p$, $q \in Q$
\item $mq\# \rightarrow p\#\#$, kde $1 \le m < n$, $p$, $q \in Q$
\item $mq\# \rightarrow pa$, kde $a \in {\Sigma}^+$, $1 \le m < n$, $p$, $q \in Q$

\end{enumerate}

\end{Def}


\begin{Alg}
Převod hlubokého zásobníkového automatu do normální formy.

\begin{list}{}{\setlength\parsep{0cm} \setlength\itemsep{0cm} \setlength\leftmargin{1em}}
   \item Vstup: $M = (Q,\Sigma,\Gamma, R, s, S, F)$ 
   \item Výstup: $M_{NF} = (Q_{NF}, \Sigma_{NF}, {\Gamma}_{NF}, R_{NF}, s_{NF},  S_{NF}, F_{NF})$ \medskip
  
  \item ${\Sigma}_{NF} := \Sigma$
  \item ${\Gamma}_{NF} :=\{\#\} \cup \Sigma$
  \item $s_{NF} := <s,S>$
  \item $S_{NF} := \#$ \medskip

  \item $k := \mathrm{max}(\{m \mid mqA \rightarrow pv \in R\}) $ \medskip

  \item Pro každé $(q,u,A) \in Q \times (\Gamma - \Sigma)^* \times (\Gamma - \Sigma)$, kde $|u| \le k$ : \medskip

  \subitem Pokud $|u| = k$, přidej do $R_{NF}$ pravidlo typu (i) :
  \subitem $k+1 <q,uA> \# \rightarrow <q,u> A$. \medskip

  \subitem Pokud $|u| < k$, přidej do $R_{NF}$ pravidlo typu (ii):
  \subitem $|uA| <q,u> A \rightarrow <q,uA> \#$. \medskip

  \item Pro každé $r : mqA \rightarrow p X_1 X_2 X_3 \dots X_j \in R$, kde $1 \le i \le j$, $X_i \in \Gamma$, a 
        pro každou dvojici $(u,v) \in (\Gamma - \Sigma)^* \times (\Gamma - \Sigma)^*$, kde $|uv| < k$ : \medskip

  \subitem Pokud $|u| = m - 1$, $|v| \le k-m$, přidej do $R_{NF}$ pravidlo typu (iii) :
  \subitem $m <q,uAv> \# \rightarrow <p,u (r : X_1 X_2 X_3 \dots X_j) v>\#$ . \medskip

  \subitem Pro každé $(i,q') \in \{1,2,3,\dots,j\} \times Q$: \medskip

  \subsubitem Pokud $i < j$, přidej do $R_{NF}$ pravidlo typu (iv) :
  \subsubitem $|u|+1 <q',u(r : X_i X_{i+1} \dots X_j)v> \# \rightarrow <q',u(r : X_i) (r: X_{i+1} \dots X_j)v> \#\# $. \medskip

  \subsubitem Pokud $X_i \in (\Gamma - \Sigma)$, přidej do $R_{NF}$ pravidlo typu (iii) :
  \subsubitem $|u|+1 <q',u (r : X_i) v> \# \rightarrow <q',u X_i v> \# $. \medskip

  \subsubitem Pokud $X_i \in \Sigma$, přidej do $R_{NF}$ pravidlo typu (v) :
  \subsubitem $|u|+1 <q',u (r : X_i) v> \# \rightarrow <q',uv> X_i $. \bigskip

  \item $Q_{NF} := \{p,q \mid mqA \rightarrow pv \in R_{NF}\} $
  \item $F_{NF} := \{<q, \epsilon> \mid  q \in F \}$


\end{list}
\end{Alg}


%%%%%%%%%%%%%%%%%%%%%%%%%%%%%%%%%%%%%%%%%%%%%%%%%%%%%%%%%%%%%%%%%%%%%%%%%%%%%%%%%
\section{Omezení počtu stavů}

% odkaz na článek on state grammars
% algoritmus

\begin{Alg}
Převod zobecněného hlubokého zásobníkového automatu s~$\epsilon$-pravidly na ekvivalentní s redukcí na tři stavy.

\begin{list}{}{\setlength\parsep{0cm} \setlength\itemsep{0cm} \setlength\leftmargin{1em}}
   \item Vstup: $M = (Q,\Sigma,\Gamma, R, s, S, F)$ 
   \item Výstup: $M_{R} = (Q_{R}, \Sigma_{R}, {\Gamma}_{R}, R_{R}, s_{R},  S_{R}, F_{R})$ \medskip
    
   \item ${\Sigma}_{R} := \Sigma$
   \item ${\Gamma}_{R} := \Sigma \cup {\Gamma}_{R}'$
   \item $Q_{R} := \{s_\alpha, s_\beta, s_\gamma \}$
   \item $R_{R} := R_{\alpha} \cup R_{\beta} \cup R_{\gamma}$
   \item $s_{R} := s_{\gamma} $
   \item $S_{R} := <start> $
   \item $F_{R} := \{s_{\alpha}, s_{\beta}\} $ \medskip

   \item Pro každé $q \in Q$, $X \in (\Gamma - \Sigma)$, $j \in \{1,2,3,\dots,|Q|\}$ přidej do ${\Gamma}_{R}'$ symboly :
   \item $<start>, <q, X>, <q, X>', <q, \#>, <q, \#>_{exp}, <j>_\alpha, <j>_\beta \}$.\medskip

   \item Nechť $Q = \{s_0, s_1, s_2, \dots,s_{|Q|}\}$, kde $s_0 = s$ je počáteční stav v~$M$. Nechť $s_{|Q|+1} = s_0$.
   \item Označme $Q_\alpha \subseteq Q$ jako množinu stavů s~lichými indexy a $Q_\beta \subseteq Q$ jako množinu stavů se sudými indexy včetně nuly.
         Pak platí $Q_\alpha \cap Q_\beta = \emptyset$ a  $Q_\alpha \cup Q_\beta = Q$.\medskip

   \item Pro každé $qA \rightarrow p b_0 B_1 b_1 B_2 b_2 \dots b_{j-1} B_{j} b_j \in R$, kde $q \in Q_\beta$, $p \in Q$, $j \in \{0,1,2,3,\dots,n\}$, $b_0,b_k \in {\Sigma}^*$ a $A, B_k \in (\Gamma - \Sigma)$ pro všechna $k \in \{1,2,\dots,j\}$. 
         Nechť $k$ je absolutní hodnota rozdílu indexů stavů $p$ a $q$. 
         Pak přidej do $R_\alpha$ pravidlo:


\begin{enumerate}
\renewcommand{\labelenumi}{(\roman{enumi})}

   \item  $s_\alpha <q, A> \rightarrow s_\alpha b_0 <q, B_1> b_1 <q, B_2> b_2 \dots b_{j-1} <q, B_j> b_j$ pokud $p = q$,
   \item jinak $s_\alpha <q, A> \rightarrow s_\beta <k>_\beta b_0 <q, B_1> b_1 <q, B_2> b_2 \dots b_{j-1} <q, B_j> b_j$. 

\suspend{enumerate}

   \item Přidej do $R_\alpha$ pravidla:

\resume{enumerate}
\renewcommand{\labelenumi}{(\roman{enumi})}

   \item $s_\alpha <s_i, X> \rightarrow s_\alpha <s_{i+1}, X>'$ pro každé $s_i \in Q_\alpha$ a každé $X \in (\Gamma - \Sigma)$,
   \item $s_\alpha <s_i, \#> \rightarrow s_\beta <s_{i+1}, \#>_{exp}$ pro každé $s_i \in Q_\alpha$,
   \item $s_\alpha <s_i, \#>_{exp} \rightarrow s_\gamma <s_{i}, \#>$ pro každé $s_i \in Q_\alpha$,
   \item $s_\alpha <q, \#> \rightarrow s_\alpha \epsilon $ pro každé $q \in Q_\beta \cap F$,
   \item $s_\alpha <1>_{\beta} \rightarrow s_\gamma \epsilon $,
   \item $s_\alpha <j>_{\beta} \rightarrow s_\alpha <j - 1 >_\alpha $ pro každé $1 < j \le |Q|$.

\suspend{enumerate}{\setlength\leftmargin{1em}}

   \item Analogicky sestroj množinu $R_\beta.$ \medskip

   \item Přidej do množiny $R_\gamma$ pravidla:

\resume{enumerate}
\renewcommand{\labelenumi}{(\roman{enumi})}

   \item $s_\gamma <start> \rightarrow s_\alpha <s_0, S> <s_0, \#>$,
   \item $s_\gamma <s, X>' \rightarrow s_\gamma <s, X> $ pro každé $s \in Q$ a každé $X \in (\Gamma - \Sigma)$,
   \item $s_\gamma <s_i, \#> \rightarrow s_\alpha <s_i, \#>$ pro každé $s_i \in Q_\alpha$ a analogicky pro $Q_\beta$,
%   \item $s_\gamma <s_i, \#> \rightarrow s_\beta <s_i, \#>$ pro každé $s_i \in Q_\beta$,
   \item $s_\gamma <s_i, \#>_{exp} \rightarrow s_\beta <s_{i}, \#>$ pro každé $s_i \in Q_\alpha$ a analogicky pro $Q_\beta$.
%   \item $s_\gamma <s_i, \#>_{exp} \rightarrow s_\alpha <s_{i}, \#>$ pro každé $s_i \in Q_\beta$,

\end{enumerate}

\end{list}
\end{Alg}




%%%%%%%%%%%%%%%%%%%%%%%%%%%%%%%%%%%%%%%%%%%%%%%%%%%%%%%%%%%%%%%%%%%%%%%%%%%%%%%%% Závěr
% Závěrečná kapitola obsahuje zhodnocení dosažených výsledků se zvlášť vyznačeným vlastním přínosem studenta. Povinně se 
% zde objeví i zhodnocení z pohledu dalšího vývoje projektu, student uvede náměty vycházející ze zkušeností s řešeným 
% projektem a uvede rovněž návaznosti na právě dokončené projekty.

%V~této práci jsem dokázala, že hluboký zásobníkový automat konečného indexu omezením počtu nevstupních symbolů neztrácí na své síle. Následně jsem ukázala ekvivalenci tohoto typu automatu s~programovými gramatikami konečného indexu. Z~toho vyplývá, že rodiny jazyků, které tento automat přijímá, tvoří nekonečnou hierarchii.



