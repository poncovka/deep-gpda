
%%%%%%%%%%%%%%%%%%%%%%%%%%%%%%%%%%%%%%%%%%%%%%%%%%%%%%%%%%%%%%%%%%%%%%%%%%%%%%% Definice
%V~této práci používám značení a pojmy z~teorie formálních jazyků zavedené například v~\cite{Meduna:Theory}. Dále vycházím z~definice maticové a programové gramatiky v~\cite{Dassow:RegulatedRewriting}.


%Pozn. Je třeba si uvědomit, že hluboký zásobníkový automat konečného indexu a gramatika konečního indexu se mají jinak definovanou svoji konečnost. Zatímco u hlubokého zásobníkového automatu nemůže nikdy dojít k tomu, že na zásobníku bude více než n nevstupních symbolů, gramatiky konečného indexu mohou expandovat řetězec s více než n neterminály. Pro každý řetězec terminálů však existuje taková sekvence derivačních kroků, která v žádném kroku neobsahuje více než n neterminálů.


%\begin{Def}
%Bezkontextová gramatika je dle \cite{doplnit} čtveřice $G = (V,T,P,S)$, kde V je úplná abeceda, T \subset V je abeceda terminálů. Nechť N = (V - T) je abeceda neterminálů, pak S \in N je počáteční symbol. P je konečná množina pravidel, která zapisujeme ve tvaru q: A \rightarrow v, kde A \in N, v \in V^* a q je označení pravidla.

%Nechť q: A \rightarrow v \in P, x,y \in V^*, pak bezkontextová gramatika G provede derivační krok z xAy do xvy podle pravidla q; píšeme xAy \Rightarrow xvy [q] nebo zjednodušeně xAy \Rightarrow xvy. Nechť q_1, q_2, \dots q_m \in P, m \ge 0, pak G může provést sekvenci kroků dle těchto pravidel; zápisem x {\Rigtarrow}^m y [q_1 q_2 \dots q_m]. Píšeme {\Rigtarrow}^+ pro libovolné m > 0 a {\Rigtarrow}^* pro m \ge 0. Gramatika G generuje jazyk L(G), pro kteý platí, že L(G) = {w \in T^* | S {\Rigtarrow}^* w}. Jazyk L je bezkontektový, právě když L = L(G).

%\end{Def}

%\begin{Def}
%Programová gramatika je čtveřice $G = (V,T,P,S)$, která vychází z bezkontektové gramatiky. P je konečná množina pravidel tvaru q:A \rigtarrow v, g(p), kde q:A \rigtarrow definujeme stejně jako v bezkotextové gramatice a q(p) je množina značení těch pravidel, která mohou být provedena v dalším derivačním kroku.

%TODO zavést sekvence derivačních kroků
%\end{Def}

% zavést index gramatiky

% zavést maticovou gramatiku

%%%%%%%%%%%%%%%%%%%%%%%%%%%%%%%%%%%%%%%%%%%%%%%%%%%%%%%%%%%%%%%%%%%%%%%%%%%%%%%%%






%%%%%%%%%%%%%%%%%%%%%%%%%%%%%%%%%%%%%%%%%%%%%%%%%%%%%%%%%%%%%%%%%%%%%%%%%%%%%%%%% Závěr
% Závěrečná kapitola obsahuje zhodnocení dosažených výsledků se zvlášť vyznačeným vlastním přínosem studenta. Povinně se 
% zde objeví i zhodnocení z pohledu dalšího vývoje projektu, student uvede náměty vycházející ze zkušeností s řešeným 
% projektem a uvede rovněž návaznosti na právě dokončené projekty.

%V~této práci jsem dokázala, že hluboký zásobníkový automat konečného indexu omezením počtu nevstupních symbolů neztrácí na své síle. Následně jsem ukázala ekvivalenci tohoto typu automatu s~programovými gramatikami konečného indexu. Z~toho vyplývá, že rodiny jazyků, které tento automat přijímá, tvoří nekonečnou hierarchii.



